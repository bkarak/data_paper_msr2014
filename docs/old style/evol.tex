\documentclass[conference]{llncs}

\usepackage{url}
\usepackage{amsmath}
\usepackage{graphicx}
\usepackage{subfigure}
\usepackage{threeparttable}
\usepackage{pdflscape}
\usepackage{array}

\begin{document}

\title{Dismal Code: Studying the Evolution of Security Bugs}

\author{Dimitris Mitropoulos \and Vassilios Karakoidas \and Panos Louridas \and Georgios Gousios \and Diomidis Spinellis }
\institute{ Department of Management Science and Technology\\
			Athens University of Economics and Business\\
      		\email{\{dimitro, bkarak, louridas, dds\}@aueb.gr}\\
			Software Engineering Research Group\\
			Delft University of Technology\\
			\email{G.Gousios@tudelft.nl}}

\maketitle

\begin{abstract}
%\boldmath
Security bugs are critical programming errors that can lead to serious
vulnerabilities in software. Such bugs may allow an attacker to take over
an application, steal data or prevent the application from working at all.
We used the projects stored in the Maven repository to study the
characteristics of security bugs individually and in relation to other software
bugs. In particular, we analyzed every project version of the repository by using a
popular static analysis tool. Then, by taking advantage of
the repository's project history and dependency data we studied
the evolution of security bugs through time. In addition, we examined
their persistence and their relationship with
a) the size of the corresponding version, and
b) other bug categories.
Our findings indicate that there is no simple rule governing the number of
security bugs as a project evolves.
In particular, we cannot say that across projects security-related
defect counts increase or decrease significantly over time.
Furthermore, security bugs are not eliminated in a
way that is particularly different from the other bugs.
In addition, the relation of security bugs with a project's size
appears to be different from the relation of the bugs
coming from other categories.
Finally, even if bugs seem to have similar behaviour,
severe security bugs seem to be unassociated with other bug categories.
\end{abstract}

\begin{keywords}
Security Bugs, Static Analysis, Software Evolution, Software
Security, Maven, FindBugs.
\end{keywords}

\section{Introduction}

A security bug is a programming error that introduces a potentially exploitable
weakness into a computer system~\cite{SSL12,TJBD11,M06}. This
weakness could lead to a security breach with unfortunate consequences in
different layers, like databases, native code, applications, libraries and
others. Despite the significant effort to detect and eliminate such
bugs~\cite{SZ12}, little attention has been paid to study them in relation to
software evolution~\cite{L96,LRWPT97,IB06,RGMA06}. 
One of the most common approaches to identify security bugs is
{\it static analysis}~\cite{CW07}. This kind of analysis involves the
inspection of the program's source or object code without executing
it. 

In this
paper we present how we used a large software ecosystem to analyse how different
types of security vulnerabilities are related to the evolving software packages.
For our research we used {\it FindBugs},\footnote{\url{http://findbugs.sourceforge.net/}}
a static analysis tool that examines bytecode to detect software bugs and has already been used in
research~\cite{AP10,HP07,HP04,HW08,SHP06}.
Specifically, we ran FindBugs on all the project
versions of all the projects that exist in the
{\it Maven Central Repository}\footnote{\url{http://search.maven.org/}}
(approximately 265GB of data---see Section~\ref{sec:data}).
Then we observed the changes that involved the security bugs and their characteristics.
This research builds upon our earlier work on the topic~\cite{MGS12}.

We chose to focus our study on security bugs rather than other types of
software defects. This is because compared to other bug categories,
failures due to security bugs have two distinct features:
they can severely affect an organization's infrastructure~\cite{SZ12}, and
they can cause significant financial damage to
an organization~\cite{TH04,BCL08,R06}. 
Specifically,
whereas a software bug can cause a software artifact to fail,
a security bug can allow a malicious user to alter the execution
of the entire application for his or her own gain.
In this case, such bugs could give rise to a wide
range of security and privacy issues, like
the access of sensitive information,
the destruction or modification of data, and
denial of service.
Moreover, security bug disclosures lead to a negative and significant change
in market value for a software vendor~\cite{TW07}.
Hence, one of the basic pursuits in every new software release should
be to mitigate such bugs.

% @dimitro: say why we did this work, not how we did it
The motivation behind our work was to validate whether programmers care for
the risk posed by security bugs when they release a new version of their software.
In addition, we wanted to investigate other critical features associated with such
vulnerabilities like the persistence of a bug;
in essence, to see whether critical bugs stay unresolved for a long time.
Also, we wanted to elaborate more on the relation of security
bugs with other bug categories.
In the same manner, we tried to examine the relationship
between the size of a project release and the number of security bugs that it contains,
knowing the that research has produced contradictory results on this
issue~\cite{BP84,SYTP85,NBZ06,GKMS00}.
Finally, we examined the Maven ecosystem as a whole from a security
perspective. Its structure gave us the opportunity to see if a project version that is a dependency to
a large number of others contains a low rate of security bugs~\cite{MW10}.
%In this case we can strengthen the relationship between Linus' Law and
%software security as Meneely et al. have already indicated~\cite{MW10}.

In this work we:
\begin{itemize}
	\item Analyze how security bugs found through static analysis
evolve over time. To achieve this, we inspected all releases of every project.
Our hypothesis is that security bugs should decrease as a project evolves,
for they form critical issues, which developers should eliminate.
%We present evidence that defect counts remain relatively stable
%across project revisions.
	\item Examine security bug persistence across releases. 
We expect that security bugs should be eliminated earlier than other bugs.
%We find that a specific defect remains unfixed on average 
%for 3 revisions.
	\item Study the relation between security bugs and a project
release's size. Our hypothesis is that security bugs are proportional to a project
release's size (defined in terms of bytecode size).
%We show that many defect types are correlated with the project's
%size (with the notable exception of high-risk security defects).
	\item Examine the correlation of security bugs with other bug categories.
Our hypothesis is that security bugs appear together with bugs that
are related with performance, coding practices, and product stability.
%We show that a code base can have defect hotspots.
%	\item The relation of software dependencies and security bugs. Our
%hypothesis is that an artifact that is a dependency of many other artifacts
%should have a low rate of security bugs. 
  %\item Make available a carefully constructed dataset and a statistical toolkit for
   % performing analysis of the Java Maven ecosystem.
\end{itemize}

In the rest of this paper we
describe the processing of our data and our experiment (Section \ref{sec:meth}),
present and discuss the results we obtained (Section \ref{sec:res}),
outline related work (Section \ref{sec:rel}),
and end up with a conclusion and directions for future work (Section \ref{sec:con}).

\section{Methodology}
\label{sec:meth}

Our experiment involved the collection of the metric results of the FindBugs
tool. Before and during the experiment, we performed a number of functional
fits on the data coming from the Maven repository, for reasons that we will describe below.

\subsection{Experiment}
\label{sec:exp}

\begin{figure}[t]
  \begin{center}
    \includegraphics[scale=0.75]{figures/arch.pdf}
  \end{center}
  \caption{The Data Processing Architecture}
  \label{fig:arch}
\end{figure}

The goal of our experiment was to retrieve all the bugs that FindBugs reports,
from all the project versions existing on the Maven repository (in the Maven
repository, versions are {\it actual} releases). The experiment involved four entities:
a number of {\it workers} (a custom Python script), a {\it task queue}
mechanism (Rabbit{\sc mq} - version 3.0.1),\footnote{\url{http://www.rabbitmq.com/}}
a {\it data repository} (Mongo{\sc db} - version 2.2),\footnote{\url{http://www.mongodb.org/}}
and the {\it code repository}, which in our case it was
the public Maven repository.

%\begin{table}
%\centering
%\begin{threeparttable}
%\caption{Experiment Entities}
%\label{tbl:exp}
%\begin{tabular}{l r r}
%\hline
%Entity & Software Package & Version\\
%\hline
%Worker & Custom Python Script & 2.7\\
%Task Queue & RabbitMQ & 3.0.1 \\
%Data Repository & MongoDB & 2.2 \\
%Code Repository & Maven Central Repository & January 2012 \\
%\hline
%\end{tabular}
%\end{threeparttable}
%\end{table}

First, we scanned the Maven repository for appropriate {\sc jar}s and created a
list that included them. We discuss the {\sc jar} selection process in the next 
section. With the {\sc jar} list at hand, we created a series of processing tasks
and added them to the task queue. Then we executed twenty five (Unix-based)
workers that checked out tasks from the queue, processed the data and stored the
results to the data repository.

A typical processing cycle of a worker included the following steps: after
the worker spawned, it requested a task from the queue. This task contained
the {\sc jar} name, which was typically a project version that was downloaded locally.
First, specific {\sc jar} metadata were calculated and stored. Such metadata included
its size, its dependencies, and a number that represented the chronological order of the
release. This order was derived from an {\sc xml} file that
accompanies every project in the Maven repository called {\it
maven-metadata.xml}. Then, FindBugs was invoked by the worker and its results were
also stored in the data repository. When the task was completed the queue
was notified and the next task was requested. This process was executed for
all the available {\sc jar}s in the task queue. A schematic representation of
the data processing architecture can be seen in Figure~\ref{fig:arch}.

\subsection{Data Provenance}
\label{sec:data}

Maven is a build automation tool used primarily for Java projects and it is
hosted by the Apache Software Foundation.\footnote{\url{http://www.apache.org/}}
To describe the software project being built, its dependencies
on other external modules, the build order, and required plug-ins Maven uses
{\sc xml}. To build a software component, it dynamically downloads Java libraries
and Maven plug-ins from the Maven central repository,
and stores them in a local cache. The repository can be updated with
new projects and also with new versions of existing projects.

Initially, we obtained a snapshot (January 2012) of the Maven repository and
handled it locally to retrieve a list of all the names of the project versions
that existed in it. A project version can be uniquely identified by the triplet:
{\it group id}, {\it artifact id} and {\it version}.
Since FindBugs analyses applications written in the Java
programming language, and the Maven repository
hosts projects from languages other than Java such as Scala, Groovy,
Clojure etc. we filtered out such projects by performing a series of checks in
the repository data and metadata.

\begin{table}[t]
\centering
\caption{Descriptive Statistics Measurements for the Maven Repository}
\label{tbl:repository}
\begin{tabular}{l r}
\hline
Measurement & Value\\
 \hline
Projects & 17,505\\
Versions (total) & 115,214\\
Min (versions per project) & 1\\
Max (versions per project) & 338\\
Mean (versions per project) & 6.58\\
Median (versions per project) & 3\\
Range (over versions) & 337\\
1$^{st}$ Quartile (over versions) & 1\\
3$^{rd}$ Quartile (over versions) & 8\\
\hline
\end{tabular}
\end{table}

In addition, we implemented a series of audits in the worker scripts that
checked if the {\sc jar}s are valid in terms of implementation. For instance,
for every {\sc jar} the worker checked if there were any {\it .class} files
before invoking FindBugs. After the project filtering, we narrowed down
our data set to 17,055 projects with 115,214 versions.
Table~\ref{tbl:repository} summarises the data set information and
provides the basic descriptive statistic measurements. The distribution of version
count among the selected projects is presented in Figure \ref{fig:version-count}.

\begin{figure*}[t]
	\centering
	\includegraphics[scale=0.45]{version_count.pdf}
	\caption{Distribution of Version Count Among Project Population}
	\label{fig:version-count}
\end{figure*}

The statistical measurements presented in Table~\ref{tbl:repository}
indicate that we have 17,505 projects and the data set's median is 3,
which means that almost 50\% (8,753 projects) of the project
population have 1 to 3 versions. In general, most projects have a few
number of versions, there are some projects with ten versions and
only a few with hundreds of versions. The maximum number of versions
for a project is 338. The 3$^{rd}$ quartile measurement also indicated
that 75\% (13,129) of the projects have a maximum of 8 versions.

\subsection{Threats to Validity}
\label{sec:threats}

A threat to the internal validity of our experiment could be the false alarms of the
FindBugs tool. False positives and negatives of static analysis tools and
how they can be reduced is an issue that has already been discussed in the literature
(see Section~\ref{sec:rel}).
In addition, reported security bugs may not be applicable to an
application's typical use context.
For instance, FindBugs could report an {\sc sql} injection vulnerability~\cite{RL12}
in an application that receives no external input.
In this particular context, this would be a false positive alarm.

Furthermore, given that our analysis is done on open-source projects
written in the Java programming language and hosted on Maven,
a threat to the external validity of our work
is the fact that our results may not be applicable to other
programming languages, ecosystems, and development cultures.

\section{Results and Analysis}
\label{sec:res}

\begin{table}[t]
\centering
\caption{Bug Categorisation According to FindBugs}
\label{tbl:bug-cat}
\begin{tabular}{l p{23em}}
\hline
Category & Description\\
\hline
Bad Practice & Violations of recommended and essential
coding practice. \\
Correctness & Involves coding misting a way that is particularly different from the other bug sakes resulting in code
that was probably not what the developer intended. \\
Experimental & Includes unsatisfied obligations. For instance,
forgetting to close a file. \\
Internationalization (i18n) & Indicates the use of non-localized methods. \\
Multi-Threaded ({\sc mt}) Correctness & Thread synchronization issues. \\
Performance & Involves inefficient memory usage allocation, usage 
of non-static classes. \\
Style & Code that is confusing, or
written in a way that leads to errors.\\
Malicious Code & Involves variables or fields exposed to classes that should
not be using them. \\
Security & Involves input validation issues, unauthorized database connections
and others. \\
\hline
\end{tabular}
\end{table}

Our findings can be analysed at two levels. First, we discuss some
primary observations concerning the security bugs of the Maven repository as a whole.
Then, we provide a comprehensive analysis of the results and highlight our key findings.

\subsection{Overview and Initial Results}
\label{sec:overview}

FindBugs separates software bugs into nine categories (see
Table~\ref{tbl:bug-cat}). Two of them involve security issues: {\it Security} and {\it
Malicious Code}. From the total number of releases, 4,353 of them contained
at least one bug coming from the first category
and 45,559 coming from the second.

\begin{table*}[t]
\centering
\caption{Number of Project Releases that Contain at Least One Severe Security Bug}
\label{tbl:sev}
\leavevmode
	\begin{tabular}{p{24em} r}
	\hline
	Bug Description & Number of Project Releases\\
 	\hline
	{\sc hrs}: {\sc http} cookie formed from untrusted input & 151\\
	{\sc hrs}: {\sc http} response splitting vulnerability & 1,579\\
	{\sc pt}: absolute path traversal in servlet  & 103\\
	{\sc pt}: relative path traversal in servlet & 57\\
	{\sc sql}: non-constant string passed to execute method on an {\sc sql} statement & 1,875\\
	{\sc sql}: a prepared statement is generated from a non-constant String & 1,486\\
	{\sc xss}: {\sc jsp} reflected cross site scripting vulnerability & 18\\
	{\sc xss}: Servlet reflected cross site scripting vulnerability in error page & 90\\
	{\sc xss}: Servlet reflected cross site scripting vulnerability & 142\\
	\hline
	\end{tabular}
\end{table*}

Our first results involve the most popular bugs in the Maven repository.
Figure~\ref{fig:bug-per} shows how software bugs are distributed among the
repository. Together with the {\it Bad Practice} bugs and the {\it Style} bugs,
security bugs (the sum of the {\it Security} and {\it Malicious Code}
categories) are the most popular in the repository ($\geq 21\%$ each).
This could be a strong indication that programmers write code
that implements the required functionality without considering its many
security aspects; an issue that has already been reported in
literature~\cite{SH09}.
% An interesting observation is that 4,908 projects did not
% have any bugs at all. These cases involved small artifacts with a minimum number of classes.
% This goes along with the relationship between the size of an artifact and it's
% bugs that is presented in the upcoming section.

\begin{figure*}[t]
	\centering
	\includegraphics[scale=0.45]{bug_percent}
	\caption{Bug Percentage in Maven Repository}
	\label{fig:bug-per} 
\end{figure*}

Another observation involves bugs that we could call {\it
severe} and they are a subset of the {\it Security} category.
Such bugs are related to vulnerabilities that appear due to the lack of user-input
validation and can lead to damaging attacks like {\sc sql} injection~\cite{RL12} and
Cross-Site Scripting~\cite{WS08}.
Note that to exploit such vulnerabilities, a malicious user does
not have to know anything about the application internals. For all the other
bugs, another program should be written to incorporate references to
mutable objects, access non-final fields, etc.
Also, as bug descriptions indicate,\footnote{\url{http://findbugs.sourceforge.net/bugDescriptions.html}}
if an application has bugs coming from the {\it Security High} category,
it might have more vulnerabilities that FindBugs doesn't report.
Table~\ref{tbl:sev} presents the number
of releases where at least one of these bugs exists. In essence, 5,501 releases
($\approx 4,77\% $), contained at
least one severe security bug. Given the fact that other projects include these
versions as their dependencies, they are automatically rendered vulnerable if
they use the code fragments that include the defects.
In the following section
we provide results separately for this subcategory which we call {\it Security High}.
The remaining bugs of the {\it Security} category
are grouped together with the bugs of the {\it Malicious Code} category
in another subcategory that we call {\it Security Low}.

{\it Linus's Law} states that ``given enough eyeballs, all bugs are shallow".
To examine Linus's Law in relation to the rate of the
security bugs~\cite{MW10}, and highlight the
{\it domino effect}~\cite{TH04} we did the
following: during the experiment we retrieved the
dependencies of every version. Based on this information we created a graph
that represented the snapshot of the Maven repository. The
nodes of the graph represented the versions and the vertices their dependencies.
The graph was not entirely accurate. For instance, if
a dependency was pointing only to a project (and not to a specific version), we chose to
select the latest version found on the repository. Also, this graph is not
complete. This is because there were missing versions.
% Practically, the corresponding {\it pom}, or {\it war} files existed on the
% repository, but the {\sc jar} file did not.
From the 565,680 vertices, 191,433
did not point to a specific version while 164,234 were pointing to missing ones.
The graph contained 80,354 nodes. Obviously, the number does not correspond to
the number of the total versions (see Section~\ref{sec:data}). This is because
some versions did not contain any information about their dependencies so they
are not represented in the graph. After creating the graph, we ran the PageRank
algorithm~\cite{BP98} on it and retrieved all PageRanks for every node. Then we
examined the security bugs of the fifty most popular nodes based on their PageRank.
Contrary to Linus's Law, thirty three of them contained bugs coming from the
{\it Security Low} subcategory, while two of them contained severe bugs.
Twenty five of them were latest versions at the time. This also highlights
the domino effect. No matter how well a programmer secures a software component
it won't matter if he or she is using vulnerable code fragments coming
from another library.

\subsection{Analysis}
\label{sec:analysis}

Here, we present our key findings concerning the evolution of security bugs.

\subsubsection{How Security Bugs Evolve Over Time}

\begin{table}[t]
    \centering
    \caption{Correlations between Version and Defects Count}
    \label{tbl:bugsperversion}
    
\begin{tabular}{lcc}
\hline \\
Category & Spearman Correlation & $p$-value \\ \hline 
Security & 0.04 & $< 0.05$\\
Malicious Code & 0.03 & $\ll 0.05$\\
Style & 0.03 & $\ll 0.05$\\
Correctness & 0.04 & $\ll 0.05$\\
Bad Practice & 0.03 & $\ll 0.05$\\
{\sc mt} Correctness & 0.09 & $\ll 0.05$\\
i18n & 0.06 & $\ll 0.05$\\
Performance & {\it (0.01) } & 0.07\\
Experimental & 0.09 & $\ll 0.05$\\
\hline \\
\end{tabular}

\end{table}

The relation between bugs and time can be traced from the number of
bugs per category in each project version. We can then calculate the
Spearman correlations between the defects count and the ordinal
version number across all projects to see if bigger versions relate to
higher or lower defect counts. The results are shown in
Table~\ref{tbl:bugsperversion}. Although the tendency is for defect
counts to increase, this tendency is extremely slight.

The zero tendency applies to all versions of all projects together. 
The situation might be different in individual projects. We therefore
performed Spearman correlations between bug counts and version
ordinals in all projects we examined. These paint a different picture
from the above table, shown in Figure~\ref{fig:bugsversionscorr}. The
spike in point zero is explained by the large number of projects for
which no correlation could be established---note that the scale is
logarithmic. Still, we can see that there were projects where a
correlation could be established, either positive or negative. The
{\it Security High} category is particularly bimodal, but this is
explained by the small number of correlations that could be
established, nine in total.

\begin{figure*}[t]
  \centering
  \includegraphics[width=\textwidth]{bugsversionscorr}
  \caption{Histograms of correlations between bug counts and version
    ordinals per project. In brackets the total population size and
    the number of no correlation instances.}
  \label{fig:bugsversionscorr}
\end{figure*}

Overall, Table~\ref{tbl:bugsperversion} and
Figure~\ref{fig:bugsversionscorr} suggest that \textbf{we cannot say that
across projects defect counts increase or decrease significantly
across time}. In individual projects, however, defect counts can have a
strong upwards or downwards tendency. There may be no such thing as a
``project'' in general, only particular projects with their own
idiosyncrasies, quality features, and coding practices.

Another take on this theme is shown in Figure~\ref{fig:bugdiffs},
which presents a histogram of the changes of different bug counts in
project versions. In most cases, a bug count does not change between
versions; but when it does change, it may change upwards or downwards.
Note also the spectacular changes of introducing or removing thousands
of defects; this may be the result of doing and undoing a pervasive
code change that falls foul of some bug identification rule.

\begin{figure*}[t]
  \centering
  \includegraphics[width=\textwidth]{bugdiffs}
  \caption{Histograms of the time, measure in versions, it takes for a bug to be closed. }
  \label{fig:bugdiffs}
\end{figure*}

\subsubsection{Persistence of Security Bugs}

To examine the relation between the persistence of different kinds of
bugs, and of security bugs in particular, we used as a persistence
indicator the number of versions a bug remains open in a project. We
grouped the persistence numbers by bug categories and then performed a
Mann-Whitney $U$~\cite{HM98} test among all bug category pairs. The results are
presented in Table~\ref{tbl:bug_persistence}. Cells in brackets show
pairs where no statistically significant difference was found.

In general, although the average number of versions bugs in different
bug categories remained opened was statistically different in many
cases, the difference is not spectacular. \textbf{In all cases a bug persists
on average between two and three versions}, with the difference being
in the decimal digits.

\begin{landscape}
  \begin{table}
    \setlength{\extrarowheight}{0.10cm}
    \caption{Bug Persistence Comparison}
    \label{tbl:bug_persistence}
    \resizebox{0.95\columnwidth}{!}{
    
\begin{tabular}{|l|>{\centering\arraybackslash}m{2.5cm}|>{\centering\arraybackslash}m{2.5cm}|>{\centering\arraybackslash}m{2.5cm}|>{\centering\arraybackslash}m{2.5cm}|>{\centering\arraybackslash}m{2.5cm}|>{\centering\arraybackslash}m{2.5cm}|>{\centering\arraybackslash}m{2.5cm}|>{\centering\arraybackslash}m{2.5cm}|}
\hline 
Security & {\it ($0.19$, $p = 0.85$\newline 2.86, 2.36\newline 265, 35026)} & $2.48$, $p < 0.05$\newline 2.86, 2.12\newline 265, 49043 & {\it ($-0.38$, $p = 0.70$\newline 2.86, 2.50\newline 265, 12905)} & $3.05$, $p < 0.01$\newline 2.86, 2.11\newline 265, 49324 & {\it ($1.22$, $p = 0.22$\newline 2.86, 2.48\newline 265, 10227)} & {\it ($-1.10$, $p = 0.27$\newline 2.86, 2.74\newline 265, 10718)} & {\it ($-0.90$, $p = 0.37$\newline 2.86, 2.65\newline 265, 23598)} & {\it ($-0.20$, $p = 0.84$\newline 2.86, 2.85\newline 265, 2686)}\\
Malicious Code &  & $20.25$, $p \ll 0.05$\newline 2.36, 2.12\newline 35026, 49043 & $-3.59$, $p \ll 0.05$\newline 2.36, 2.50\newline 35026, 12905 & $25.16$, $p \ll 0.05$\newline 2.36, 2.11\newline 35026, 49324 & $5.58$, $p \ll 0.05$\newline 2.36, 2.48\newline 35026, 10227 & $-7.55$, $p \ll 0.05$\newline 2.36, 2.74\newline 35026, 10718 & $-8.20$, $p \ll 0.05$\newline 2.36, 2.65\newline 35026, 23598 & {\it ($-1.39$, $p = 0.16$\newline 2.36, 2.85\newline 35026, 2686)}\\
Style &  &  & $-17.96$, $p \ll 0.05$\newline 2.12, 2.50\newline 49043, 12905 & $5.66$, $p \ll 0.05$\newline 2.12, 2.11\newline 49043, 49324 & $-6.84$, $p \ll 0.05$\newline 2.12, 2.48\newline 49043, 10227 & $-20.61$, $p \ll 0.05$\newline 2.12, 2.74\newline 49043, 10718 & $-26.18$, $p \ll 0.05$\newline 2.12, 2.65\newline 49043, 23598 & $-8.30$, $p \ll 0.05$\newline 2.12, 2.85\newline 49043, 2686\\
Correctness &  &  &  & $21.38$, $p \ll 0.05$\newline 2.50, 2.11\newline 12905, 49324 & $7.44$, $p \ll 0.05$\newline 2.50, 2.48\newline 12905, 10227 & $-3.57$, $p \ll 0.05$\newline 2.50, 2.74\newline 12905, 10718 & $-2.91$, $p < 0.01$\newline 2.50, 2.65\newline 12905, 23598 & {\it ($0.40$, $p = 0.69$\newline 2.50, 2.85\newline 12905, 2686)}\\
Bad Practice &  &  &  &  & $-10.02$, $p \ll 0.05$\newline 2.11, 2.48\newline 49324, 10227 & $-23.63$, $p \ll 0.05$\newline 2.11, 2.74\newline 49324, 10718 & $-30.32$, $p \ll 0.05$\newline 2.11, 2.65\newline 49324, 23598 & $-9.98$, $p \ll 0.05$\newline 2.11, 2.85\newline 49324, 2686\\
{\sc mt} Correctness &  &  &  &  &  & $-10.17$, $p \ll 0.05$\newline 2.48, 2.74\newline 10227, 10718 & $-10.83$, $p \ll 0.05$\newline 2.48, 2.65\newline 10227, 23598 & $-4.03$, $p \ll 0.05$\newline 2.48, 2.85\newline 10227, 2686\\
i18n &  &  &  &  &  &  & {\it ($1.29$, $p = 0.20$\newline 2.74, 2.65\newline 10718, 23598)} & $2.46$, $p < 0.05$\newline 2.74, 2.85\newline 10718, 2686\\
Performance &  &  &  &  &  &  &  & {\it ($1.92$, $p = 0.05$\newline 2.65, 2.85\newline 23598, 2686)}\\
\hline
\multicolumn{1}{l|}{}  &  &  &  &  &  &  &  &  \\
\cline{1-1}
\multicolumn{2}{l|}{Malicious Code}  &  &  &  &  &  &  &  \\
\cline{1-2}
\multicolumn{3}{l|}{Style}  &  &  &  &  &  &  \\
\cline{1-3}
\multicolumn{4}{l|}{Correctness}  &  &  &  &  &  \\
\cline{1-4}
\multicolumn{5}{l|}{Bad Practice}  &  &  &  &  \\
\cline{1-5}
\multicolumn{6}{l|}{MT Correctness}  &  &  &  \\
\cline{1-6}
\multicolumn{7}{l|}{i18n}  &  &  \\
\cline{1-7}
\multicolumn{8}{l|}{Performance}  &  \\
\cline{1-8}
\multicolumn{9}{l|}{Experimental}  \\
\cline{1-9}
\hline
\end{tabular}
}\\
    The matrix presents pairwise Mann-Whitney $U$ test results
    between the different bug categories. Each cell contains the test
    result (the value of $U$), the $p$-value, the average for each
    category and the sample size for each category. Cells in brackets show
    pairs where no statistically significant difference was found.
  \end{table}
\end{landscape}

\subsubsection{The Relation of Defects with the size of a {\sc jar}}

We explored the relation between defects with the size of a project
version, measured by the size of its {\sc jar} file by carrying out
correlation tests between the size and the defect counts for each
project and version. The results, all statistically significant ($p
\ll 0.05$) can be seen in Table~\ref{tbl:jarsizecorr}.  \textbf{The {\it
  Security High} category stands out by having a remarkably lower
effect than the other categories}, even {\it Security Low} that nearly
tops the list.

\begin{table}[t]
    \centering
    \caption{Correlations between {\sc jar} size and Defects Count}
    \label{tbl:jarsizecorr}
    
\begin{tabular}{lcc}
\hline \\
Category & Spearman Correlation & $p$-value \\ \hline 
Security High & 0.19 & $\ll 0.05$\\
Security Low & 0.65 & $\ll 0.05$\\
Style & 0.68 & $\ll 0.05$\\
Correctness & 0.51 & $\ll 0.05$\\
Bad Practice & 0.67 & $\ll 0.05$\\
MT Correctness & 0.51 & $\ll 0.05$\\
i18n & 0.53 & $\ll 0.05$\\
Performance & 0.63 & $\ll 0.05$\\
Experimental & 0.36 & $\ll 0.05$\\
\hline \\
\end{tabular}

\end{table}

\subsubsection{Security Bugs {\sc vs} Other Bug Categories}

To see whether bugs flock together we performed pairwise correlations
between all bug categories. We calculated the correlations between the
number of distinct bugs that appeared in a project throughout its
lifetime. The results can be seen in Table~\ref{tbl:corrmatrix} and
Figure~\ref{fig:corrplot}. We found significant, but not always
strong, correlations between all pairs. In general, the {\it Security
  High} category showed the weakest correlations with the other
categories. Our results show that in general \textbf{bugs do flock together}.
We do not find projects with only a certain kind of bug; bugs come
upon projects in swarms of different kinds. Bugs of the {\it Security
  High} category, though, are different: they are either not
associated with other bugs, or only weakly so. Perhaps it takes a
special kind of blunder to make it a security hazard.

\begin{table}[t]
    \centering
    \caption{Correlation matrix}
    \label{tbl:corrmatrix}
    
\begin{tabular}{lccccccccc}
\hline
Security High & 1.00 & 0.21 & 0.22 & 0.25 & 0.18 & 0.34 & 0.21 & 0.31 & 0.33\\
Security Low & 0.21 & 1.00 & 0.63 & 0.63 & 0.56 & 0.56 & 0.62 & 0.60 & 0.47\\
Style & 0.22 & 0.63 & 1.00 & 0.68 & 0.62 & 0.63 & 0.66 & 0.58 & 0.46\\
Bad Practice & 0.25 & 0.63 & 0.68 & 1.00 & 0.54 & 0.58 & 0.64 & 0.63 & 0.51\\
Correctness & 0.18 & 0.56 & 0.62 & 0.54 & 1.00 & 0.56 & 0.55 & 0.52 & 0.52\\
MT Correctness & 0.34 & 0.56 & 0.63 & 0.58 & 0.56 & 1.00 & 0.56 & 0.56 & 0.49\\
Performance & 0.21 & 0.62 & 0.66 & 0.64 & 0.55 & 0.56 & 1.00 & 0.60 & 0.52\\
i18n & 0.31 & 0.60 & 0.58 & 0.63 & 0.52 & 0.56 & 0.60 & 1.00 & 0.60\\
Experimental & 0.33 & 0.47 & 0.46 & 0.51 & 0.52 & 0.49 & 0.52 & 0.60 & 1.00\\
\hline
\end{tabular}

    \\
    \small The matrix presents pairwise correlations between the
        different bug categories.
\end{table}

\begin{figure*}[t]
  \centering
  \includegraphics[scale=0.6]{corrplot.pdf}
  \caption{Correlation matrix plot for bug categories}
  \label{fig:corrplot}
\end{figure*}

\section{Related Work}
\label{sec:rel}

There are numerous methods for mining software repositories in the context
of software evolution~\cite{KCM07}. In this section we focus on the ones
that highlight the relationship between software bugs and evolution and try to
extract useful conclusions. The key idea behind this is
similar to ours: the combination of information like bug descriptions,
documentation and others with the information retrieved from either the source
or object code of a project version.

{\it Refactoring identification} through software evolution is an approach used to
relate refactorings with software bugs. Wei{\ss}gerber et al. found that a high
ratio of refactorings is usually followed by an increasing ratio of bug
reports~\cite{WD06}. In addition, they indicated that software bugs are sometimes introduced
after an incomplete refactoring~\cite{GW05}.
Ratzinger et al.~\cite{RSG08} showed that the number of bugs decreases, when the number of
refactorings increases. Finally, Kim M. et al.~\cite{KCK11} indicated that {\sc api}-level
refactorings aid bug fixes.

{\it Micro patterns}, proposed by Kim et al.~\cite{KPW06}
detect bug-prone patterns among source code. Micro patterns describe programming
idioms like inheritance, data management, immutability and others. The approach
involves the examination of all revisions of three open-source projects to extract bug
introduction rates for each pattern. Gil et al.~\cite{GM05} analysed the
prevalence of micro patterns across five Sun {\sc jdk} versions to conclude that
pattern prevalence tends to be the same in software collections.

{\it Querying techniques} are used to answer a broad range of questions
regarding the evolution history of a project~\cite{HG05}. Bhattacharya et
al.~\cite{BN11,B11} proposed a framework that is based upon
recursively enumerable languages. The framework can correlate software
bugs with developers in various ways. For instance, return the list of
bugs fixed by a specific developer. Fischer et al.~\cite{FPG03} proposed
an approach for populating a release history database that combines code
information with bug tracking data. In this way, a developer can couple files
that contain common bugs, estimate code maturity with respect to the bugs
and others. The ``Ultimate Debian Database''~\cite{NZ10} is an {\sc sql}-based
framework that integrates information about the Debian project from various
sources to answer queries related to software bugs and source code.

D'Ambros et al. have used {\it bug history analysis} to detect
the critical components of a project~\cite{D08}. This is done by using an
evolutionary meta-model~\cite{DL08}. The same approach was
also used by Zimmermann et al.~\cite{ZNA08} to check the correlation
of bugs with software properties like code complexity, process quality and others
and predict future properties.

The evolution of software artifacts has also been analysed to {\it reduce the false
alarms} of the various static analysis tools. To achieve this, Spacco et
al.~\cite{SHP06} introduced {\it pairing} and {\it warning signatures}. In the
former, they tried to pair sets of bugs between versions in order to find
similar patterns. In the latter, they computed a signature for every bug. This
signature contained elements like the name of the class where the bug was found,
the method and others. Then they searched for similar signatures between
versions. In their research they also studied the evolution of 116 sequential
builds of the Sun Java Sevelopment Kit ({\sc jdk}). Their findings indicated that
high priority bugs are fixed over time. To improve the precision of bug
detection, Kim et al.~\cite{KE07b,KE07} proposed a history-based warning
prioritization algorithm by mining the history of bug-fixes of three
different projects. The algorithm was based on the following hypothesis: if a
bug that belongs to a specific category is eliminated by a code change, then
this warning category is important. Working towards the same direction, Heckman
et al.~\cite{HW09,HW08} have introduced benchmarks that use specific
correlation algorithms and classification techniques to evaluate alert
prioritization approaches.

Lu et al.~\cite{LAAL13} studied the {\it evolution of file-system code}.
Specifically, they analysed the changes of Linux file-system patches to extract
bug patterns and categorize bugs based on their impact. Their findings
indicated that the number of file-system bugs does not die down over time. By
categorizing bugs they also showed the frequency of specific bugs in specific
components.

Completing the above approaches, our work focuses on the subset of security bugs.
Focusing on such bugs is not a new idea. Massacci et al.~\cite{MNN11} observed
the evolution of software defects by examining six major versions of Firefox.
To achieve this they created a database schema that contained information
coming from the ``Mozilla Firefox-related Security Advisories'' ({\sc mfsa})
list,\footnote{\url{http://www.mozilla.org/projects/security/known-vulnerabilities.html}}
Bugzilla entries and others. Their findings indicated that security bugs are
persistent over time. They also showed that there are many web users that use
old versions of Firefox, meaning that old attacks will continue to work.
Zaman et al.~\cite{ZAH11} focused again on Firefox to study the relation of
security bugs with performance bugs. This was also done by analysing the project's
Bugzilla. Their research presented evidence that security bugs require more experienced developers
to be fixed. In addition, they suggested that security bug fixes are more complex than the
fixes of performance and other bugs.
Shahzad et al.~\cite{SSL12} analysed large sets of vulnerability data-sets to observe
various features of the vulnerabilities that they considered critical. Such features
were the functionality and the criticality of the defects. Their analysis
included the observation of vulnerability disclosures, the behavior of
hackers in releasing exploits for vulnerabilities, patching and others. In
their findings they highlighted the most exploited defects and showed that
the percentage of remotely exploitable vulnerabilities has gradually increased
over the years.

\section{Conclusions and Future Work}
\label{sec:con}

In this work we analyzed a large software repository to examine the
behavior of security bugs. Specifically, we analysed more than 260GB
of interdependent project versions to see how security bugs evolve
over time, their persistence, their relation with other bug
categories, and their relationship with size in terms of bytecode.

Our primary hypothesis was that security bugs, and especially severe
ones, would be corrected as projects evolve. We found that, although
bugs do close over time in particular projects, we do not have an indication that across
projects they decrease as projects mature. Moreover, defect counts may
increase, as well as decrease in time. Contrary to our second research hypothesis,
we found that security
bugs are not eliminated in a way that is particularly different from the other bugs.
Also, having an average of two to three versions persistence in a sample where 60\%
of the projects have three versions, is not a positive result especially in the
case of the {\it Security High} bugs. Concerning the relation between severe security bugs and a project's size
we showed that they are not proportionally related. Furthermore, the
pairwise correlations between all categories indicated that even though all the other
categories are related, severe bugs do not appear together with the other bugs.
Also, it is interesting to see that security bugs were one of the top two
bug categories existing in a large ecosystem. Finally, we highlighted the
domino effect, and showed evidence that indicates that Linus's Law does not
apply in the case of the security bugs.

Contrary to the approaches that examine all versions
formed after every change that has been committed to the repository,
our observations are made from a different perspective.
The versions examined in this work were actual releases
of the projects. As a result we do not have an indication of how many changes
have been made between the releases.
In essence, these {\sc jar}s were the ones that were or still are,
out there in the wild, being used either as applications,
or dependencies of others.

Furthermore, the fact that projects have their own idiosyncrasies concerning
security bugs, could help us answer questions like: what are the common
characteristics of the projects where security bugs increase over time?

Future work on our approach could also involve the observation of other ecosystems, that
serve different languages, in the same manner such as, Python's PyPY
(Python Package Index),\footnote{\url{https://pypi.python.org/pypi}}
Perl's {\sc cpan} (Comprehensive Perl Archive Network),\footnote{\url{http://www.cpan.org/}}
and Ruby's RubyGems.\footnote{\url{http://rubygems.org/}}

Our data and code will be publicly available upon publication.

\section*{Acknowledgements}

This research has been co-financed by the European Union (European Social Fund
--– {\sc esf}) and Greek national funds through the Operational Program
``Education and Lifelong Learning'' of the National Strategic Reference Framework –
Research Funding Program: Heracleitus II. Investing in knowledge society
through the {\sc esf}.

The project has been also co-financed by the European Regional Development Fund ({\sc erdf})
and national funds and is a part of the Operational Programme ``Competitiveness \&
Entrepreneurship" ({\sc opce} II), Measure ``{\sc cooperation}" (Action I).

\bibliographystyle{splncs}
\bibliography{evol} 

\end{document}


